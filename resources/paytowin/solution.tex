我们可以把这个问题反过来想:手中初始是数字N最终变为0。可以采取以下的操作:

\begin{enumerate}
    \item 如果\( x \)能被2整除,则用\(\frac{x}{2} \)替代\(x\) 并花费A枚筹码
    \item 如果\( x \)能被3整除,则用\(\frac{x}{3} \)替代\(x\) 并花费B枚筹码
    \item 如果\( x \)能被5整除,则用\(\frac{x}{5} \)替代\(x\) 并花费C枚筹码
    \item 用\(x+1\) 或\(x-1\)替代\(x\) ,花费D枚筹码 
\end{enumerate}

一种方法是,我们全用花费D的方法,这样从N到0的花费为\(ND\).另外,我们可以采用这样的策略:
\[ N \stackrel{D}{\longmapsto}...\stackrel{D}{\longmapsto}ky \stackrel{k}{\longmapsto} y\]
其中 \( k \in \{2,3,5\}\).
需要说明的是 
\( y \in \{ \lfloor \frac{N}{k} \rfloor, \lceil \frac{N}{k} \rceil\}\).
事实上,如果 \(y<\lfloor \frac{N}{k} \rfloor\)(同样的理由对\(y>\lceil \frac{N}{k} \rceil \)也适用),就有如下的变换序列,比上述序列的花费更低
\[ N \stackrel{D}{\longmapsto}...\stackrel{D}{\longmapsto} k\lfloor \frac{N}{k} \rfloor \stackrel{k}{\longmapsto}  \lfloor \frac{N}{k} \rfloor \stackrel{D}{\longmapsto}... \stackrel{k}{\longmapsto} y\]

所以,我们的最优策略是直接用\(\pm1\)的方式由N到0,或到某个
\( \{ \lfloor \frac{N}{k} \rfloor, \lceil \frac{N}{k} \rceil\}\)
\((k \in \{2,3,5\})\)。
特别地,如果记\( f(n) \)为从\(N\)到0的最小花费,那么如果我们知道了\(f( \lfloor \frac{N}{k} \rfloor)\)和\( f( \lceil \frac{N}{k} \rceil)\),那么\(f(n)\)便可容易计算出。

上述讨论可以用动态规划方法实现,但是还需要解决从N到0有多少状态的问题。可以证明,可达的数字将由以下两式限定

\[ \lfloor \frac{N}{2^a3^b5^c} \rfloor , \lceil \frac{N}{2^a3^b5^c} \rceil \]

由于\( 0 \leq a \leq 60, 0 \leq b \leq 40 , 0 \leq c \leq 30 \)
(不然分母将大于分子),可达的数字小于\(2\times 61 \times 41 \times 31= 155062\)
(为了Accepted,程序要保证\(f(n)\)总是在n的可达数字以内)



\lstinputlisting[language=C++,numbers=left,numberstyle=\tiny,frame=shadowbox,keywordstyle=\color{blue!70}]{resources/paytowin/Main.cpp}
\begin{Parallel}[v]{0.5\textwidth}{0.45\textwidth}
    \ParallelLText{
        $A_i$ はソートして昇順にします.このとき,\[ \sum_{k=1}^{N-1}{A_{N- \lfloor k/2 \rfloor}} \] が答えになることを以下に示します.

    }
    \ParallelRText{
        将A进行升序排序后,输出\[ \sum_{k=1}^{N-1}{A_{N- \lfloor k/2 \rfloor}} \]即是本题答案。接下来将给出证明
    }
    \ParallelPar

    \ParallelLText{
        最大値を証明するには,「1. この値にできる」「2. これより良い値にはできない」をそれぞれ示すのが定石です.ここでもこの方針を採用します.
    }
    \ParallelRText{
        为证明此为满意度的最大值,需要证明:1.这是可以达到的数值,2.没有比这个值更大的其它值。接下来将分别对这两点作出证明。
    }
    \ParallelPar
\end{Parallel}

\bigskip


\begin{Parallel}[v]{0.5\textwidth}{0.45\textwidth}
    \ParallelLText{

\textbf{\large どうすれば「この値にできる」?}
    }
    \ParallelRText{
\textbf{\large 为什么这个值可以达到}
    }
    \ParallelPar
\ParallelLText{
    フレンドリーさが高い順に移動させます.$i + 1$ 番目に輪に加わった人の両隣に $2i + 1$, $2i + 2$ 番目
    の人を入れるようにすると,「まだ両隣挿入の対象になっていない人 $A $の両隣には,$A$ より早く来た
    人がいる」状態を常に保てることから,上の値を達成できることが言えます.
}
\ParallelRText{
友善度从高到低排序之后,第$i+1$个会加入第$2i+1$,$2i+2$个人之间,两人之间
}
\ParallelPar
\end{Parallel}

\bigskip

\begin{Parallel}[v]{0.5\textwidth}{0.45\textwidth}
    \ParallelLText{
\textbf{\large どうして「これより良い値にはできない」?}
    }
    \ParallelRText{
\textbf{\large 为什么没有比这个值更大的?}

    }
    \ParallelPar
    \ParallelLText{
        まず,フレンドリーさが高い順に移動させる状況のみを考えればよいことを示します.i 番目に移
        動する人のフレンドリーさを ai として a1 ≥ a2 ≥ · · · ≥ ak < ak+1 なる k があったとき,
    }
    \ParallelRText{

    }
    \ParallelPar
        
    \ParallelLText{
        したがってバブルソートの要領で,移動の順番をフレンドリーさが高い順に並べ替えても損しませ
ん.ここからはそのような状況だけを考えることにします.

    }
    \ParallelRText{

    }
    \ParallelPar

    \ParallelLText{
        $A_i$が心地よさに寄与するのは高々 1 回です.それ以外の人はどうでしょう? $A_k$ の左隣に割り込
        みが起きたら,それ以降は $A_k$ の左隣に $A_k$ よりも後に来た人がいる状態が保たれます.右隣も同様
        のことが言えるので,結局 $A_k$ の寄与は 2 回までです.
    }
    \ParallelRText{

    }
    \ParallelPar

    \ParallelLText{
        したがってバブルソートの要領で,移動の順番をフレンドリーさが高い順に並べ替えても損しませ
ん.ここからはそのような状況だけを考えることにします.

    }
    \ParallelRText{

    }
    \ParallelPar



\end{Parallel}


\begin{Parallel}[v]{0.5\textwidth}{0.45\textwidth}
    \ParallelLText{
        この解説では,0-based index を用います.次のような DP を考えます.


        DP[i] = 次の条件を満たす整数 t の集合.
– ラウンド i(0-based) の直前で x = t であるとき,ここからゲームを続けて,最終的に x = 0 になる.

    }
    \ParallelRText{
        考虑数组DP(下标从0计),\( DP[i] \)的值是满足下列整数t的集合

        前i轮中x=t,游戏以x=0结束

    }
    \ParallelPar
    \ParallelLText{

この DP を後ろから埋めていきます.まず,\( DP[N] = \{0\} \) です.

次に DP[i] の遷移を考えます.\( S_i = 0 \) の場合は簡単で,


    }
    \ParallelRText{
        这个DP从N开始向前计算。首先初始时\( DP[N]=\{0\} \)

        接下来考虑DP的转移方程。\(S_i=0\)的情况比较简单:

    }
    \ParallelPar
\end{Parallel}
\[
DP[i] = DP[i + 1] \cup \{v \oplus A_i |v \in DP[i + 1]\}
\]

\begin{Parallel}[v]{0.5\textwidth}{0.45\textwidth}
    \ParallelLText{
です.


つぎに \( S_i = 1 \)の場合です.まず  \( Ai \in DP[i + 1] \) の場合を考えます.ラウンド i の時点で \( x \notin DP[i + 1] \) 
なら,何もしなければよいです.逆に,\( x \in DP[i + 1]\) の場合,\( x \oplus Ai \in DP[i + 1] \) なので,操作をしても意
味がありません.よって DP[i] = DP[i + 1] となります.


    }
    \ParallelRText{
对于\( S_i = 1 \)的情况,先看 \( Ai \in DP[i + 1] \) 时,ラウンド i の時点で \( x \notin DP[i + 1] \) なら,什么都可以。而\( x \in DP[i + 1]\) 时,\( x \oplus Ai \in DP[i + 1] \) なので就没有意义.所以直接 \( DP[i] = DP[i + 1] \) 就可以.
    }
    \ParallelPar
    \ParallelLText{
\( Ai \notin DP[i + 1] \) の場合,どんな x に対しても,x または \( x \oplus A_i \) の少なくとも一方は DP[i + 1] に含まれ
ません.よって,DP[i] = ∅ です.

    }
    \ParallelRText{

    }
    \ParallelPar
    \ParallelLText{

この DP は,xor の基底を管理することで,効率的に計算することができます.
計算量は 1 ケースあたり \( O(N \log(\max(A_i))) \) になります.
    }
    \ParallelRText{
        DP数组在XOR运行下可以比较高效完成,每个case可以在
        \( O(N \log(\max(A_i))) \)
        完成
    }
    \ParallelPar
\end{Parallel}